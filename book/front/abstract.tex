% Abstract
\chapter*{Abstract}
\addcontentsline{toc}{chapter}{Abstract}

This document presents the complete technical architecture of an autonomous voice-controlled robotic platform built on commodity hardware. Unlike academic prototypes, this system was engineered for real-world deployment, emphasizing reliability, modularity, and measurable performance.

\section*{The Challenge}
Building a robot that can simultaneously listen, see, think, and act requires solving difficult resource contention problems. Most hobby-grade implementations use a single-threaded ``while True'' loop that freezes the moment any subsystem demands computation.

\section*{Our Approach}
We implemented a \textbf{Service-Oriented Architecture (SOA)} on a Raspberry Pi 4, using ZeroMQ for inter-process communication. Each sensory modality (audio, vision, sensors) runs as an independent service, publishing events to a central orchestrator.

Key innovations include:
\begin{itemize}[noitemsep]
    \item \textbf{Phase-Driven State Machine}: A 17-transition FSM that prevents illegal state combinations.
    \item \textbf{Dual-Brain Architecture}: High-level cognition on Pi, low-level reflexes on ESP32.
    \item \textbf{Hardware Interlocks}: The ESP32 can override Pi commands to prevent collisions.
    \item \textbf{Latest-Frame Grabber}: A threading pattern that eliminates camera buffer lag.
\end{itemize}

\section*{Result}
The system achieves a \textbf{2.8-second end-to-end latency} from voice command to wheel motion, operating reliably at 2.0 GHz with thermal stability below 60°C.

\vspace{1cm}
\textbf{Keywords}: Embedded AI, ZeroMQ, Raspberry Pi, ESP32, Voice Assistant, Computer Vision, YOLO
