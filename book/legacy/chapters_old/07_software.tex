\chapter{Software Implementation}

\section{Directory Structure}
The codebase is organized by "Biological Function":
\begin{itemize}
    \item \texttt{src/core/}: The Nervous System (Orchestrator, IPC, Config).
    \item \texttt{src/audio/}: The Ears & Mouth (STT, TTS).
    \item \texttt{src/vision/}: The Eyes (YOLO, Camera).
    \item \texttt{src/uart/}: The Spinal Cord (Serial Bridge).
    \item \texttt{src/ui/}: The Face (Display).
\end{itemize}

\section{The Orchestrator Logic}
The `Orchestrator` class is a **Finite State Machine (FSM)**.
It subscribes to \textit{all} sensory inputs but only reacts based on its current state.
\begin{itemize}
    \item If \texttt{state == LISTENING}: Ignore Vision, prioritize Audio.
    \item If \texttt{state == NAVIGATING}: Prioritize Vision/Sonar, ignore Audio (mostly).
\end{itemize}

\section{Extensibility}
To add a new sensor (e.g., Lidar):
1. Create `src/lidar/runner.py`.
2. Connect to ZMQ.
3. Publish to `topic.lidar`.
4. The Orchestrator can now subscribe to it without changing the core IPC code.
