\chapter{Codebase Organization}

The repository structure reflects the Service-Oriented Architecture.

\section{Root Topology}
\begin{itemize}
    \item \texttt{src/}: Source code for all agents/services.
    \item \texttt{config/}: YAML configuration files.
    \item \texttt{scripts/}: Build and maintenance scripts (`recreate\_venvs.sh`).
    \item \texttt{systemd/}: Service unit files for deployment.
    \item \texttt{tools/}: Standalone diagnostic scripts.
\end{itemize}

\section{Source Map (\texttt{src/})}
\begin{itemize}
    \item \texttt{src/core/}: Shared libraries (`ipc.py`, `config\_loader.py`, `orchestrator.py`).
    \item \texttt{src/audio/}: Unified audio pipeline.
    \item \texttt{src/vision/}: YOLO detector and pipeline.
    \item \texttt{src/llm/}: Gateway to Cloud Gemini.
    \item \texttt{src/uart/}: Serial bridge to specific hardware.
    \item \texttt{src/stt/}, \texttt{src/tts/}: Engines for speech processing.
\end{itemize}

\section{Navigation Strategy}
Developers should start at \texttt{src/core/orchestrator.py} to understand the system state machine, then trace individual topics (e.g., `TOPIC\_STT`) to their respective producers (`src/audio/unified\_voice\_pipeline.py`).
