\chapter{Design Decisions Driven by Physics}

\section{The 2.0 GHz Constraint}
The decision to stay on Python (vs C++) is vindicated by the CPU Overclock.
At 2.0 GHz, Python's overhead is masked enough to allow "Soft Real-Time" performance.
If we were restricted to a Pi 3 (1.2 GHz), the Vision Pipeline (3.5 FPS) would drop to <1 FPS, making the system unusable.

\section{Memory Abundance Strategy}
With **7.6 GB** of available RAM, we chose **Isolation (Venvs)** over **Efficiency (Shared Libs)**.
We waste ~500 MB on duplicate libraries (numpy installed 3 times).
\textbf{Payoff}: Zero dependency conflicts. On a 1GB board (Pi Zero 2W), this architecture would crash immediately (OOM).

\section{Cloud Dependency}
The 8.15s network blocking time is the price paid for using Gemini.
We utilize the "Thick Client" model:
\begin{itemize}
    \item \textbf{Thick}: Vision and STT are local.
    \item \textbf{Thin}: Brain is remote.
\end{itemize}
This balances Privacy (Voice/Images don't strictly need to leave, though currently Gemini receives text) with Capability.
