\chapter{The Nervous System}

\section{The Compute Core}
The central intelligence is a **Raspberry Pi 4 Model B (8GB RAM)**.
Standard industrial systems often use NVIDIA Jetson, but we chose the Pi to demonstrate that **Optimization beats Raw Power**.

\subsection{Overclocking Strategy}
To compensate for the lack of a TPU (Tensor Processing Unit), we pushed the Cortex-A72 CPU beyond its stock limits.
\begin{itemize}
    \item \textbf{Stock Clock}: 1.50 GHz
    \item \textbf{Our Clock}: \textbf{2.00 GHz} (`arm_freq=2000`)
    \item \textbf{Voltage}: Overvolt 6 (`over_voltage=6`)
    \item \textbf{Cooling}: Custom Metal Heat Sink Case with Twin Active Fans.
    \item \textbf{Result}: ~33\% boost in floating-point ops per second (FLOPS), critical for Faster-Whisper and ONNX Runtime.
\end{itemize}

\section{The Messaging Bus (IPC)}
We rejected ROS (Robot Operating System) due to its heavy JVM/Python-overhead on startup. Instead, we architected a lightweight **ZeroMQ (ZMQ)** bus.

\begin{figure}[h]
    \centering
    % Placeholder for IPC Diagram
    \begin{verbatim}
    [Vision] --(pub:6010)--> [ SUB: Orchestrator :PUB ] --(pub:6011)--> [Sensors]
    [Audio ] --(pub:6010)--> [                        ] --(pub:6011)--> [Display]
    \end{verbatim}
    \caption{ZMQ Pub/Sub Topology}
\end{figure}

\subsection{Topic Topology}
\begin{itemize}
    \item \texttt{visn.detections}: JSON payload of YOLOv11 bounding boxes.
    \item \texttt{stt.result}: Transcribed user speech.
    \item \texttt{llm.response}: Actionable JSON instructions from the Brain.
\end{itemize}
