\chapter{Orchestration \& Failure Handling}

\section{Service Boot Order}
Our `systemd-analyze` data revealed a crucial race condition managed by systemd.
\begin{verbatim}
NetworkManager (8.15s) --------------------> [Online]
                                                |
Orchestrator (1.5s) --> [Waiting for IPC] --> [Connects]
Vision (2.5s) --------> [Warming Up ONNX] --> [Ready]
\end{verbatim}

\section{The "Zombie" Risk}
The Service Memory Audit revealed `voice-pipeline` holding **239 MB** of RAM.
If this process crashes and fails to release the ZMQ socket (due to missing `LINGER`), the 240MB is freed by the OS, but the **TCP Port 6010** stays in `TIME_WAIT` state for 60 seconds.
\textbf{Correction}: The Orchestrator uses `bind()` while Agents `connect()`. This means Agents can crash safely, but if the Orchestrator crashes, the Bus itself goes down for ~60s unless `SO_REUSEADDR` is used.

\section{Watchdog Implementation}
There is currently **no hardware watchdog** verified.
\begin{itemize}
    \item \textbf{Risk}: System freeze (Kernel Panic).
    \item \textbf{Mitigation}: `systemd` handles user-space crashes (`Restart=always`), but cannot save a frozen kernel.
    \item \textbf{Recommendation}: Enable the Broadcom Hardware Watchdog via `/dev/watchdog`.
\end{itemize}
